\documentclass{article}
\usepackage[utf8]{inputenc}
\usepackage{amsmath}
\usepackage[greek,english]{babel}
\usepackage{alphabeta}

\title{%
   \normalsize \textsc{Πανεπιστήμιο Ιωαννίνων, Τμήμα Πληροφορικής και Τηλεπικοινωνίων} \\ [1em]
   \normalsize \textsc{Αλγόριθμοι και Πολυπλοκότητα} \\ [1em]
   \vspace{0.5cm}
   \hrule
   \vspace{0.4cm}
   \huge Job shop scheduling problem
   \vspace{0.5cm}
   \hrule
   \vspace{0.5cm}
}

\author{%
   \normalfont
   Κωνσταντίνος Ράντος\\
   \texttt{int02565@uoi.gr}  
}

\date{}

\begin{document}
\maketitle
\vspace{0.4cm}



\section{Διάφορα προβλήματα χρονοπρογραμμάτισμου}
\vspace{0.2cm}

\subsection {Permutation Flow-Shop Scheduling Problem (PFSSP) }
Σε αυτό το πρόβλημα, υπάρχουν πολλαπλές μηχανές και εργασίες. Κάθε εργασία πρέπει να εκτελεστεί σε κάθε μηχανή σε συγκεκριμένη σειρά. Η κύρια περιοριστική συνθήκη είναι ότι η σειρά των εργασιών πρέπει να είναι ίδια σε όλες τις μηχανές.Αν έχουμε n μηχανές τότε μπορούμε να έχουμε n! λύσεις.

\subsection{Flow-Shop Scheduling Problem (FSSP)}
Αυτό το πρόβλημα αφορά επίησης πολλαπλές μηχανές και εργασίες,οι οποίες πρέπει να εκτελεστούν σε κάθε μηχανή σε συγκεκριμένη σειρά. Διαφέρει από το PFSSP καθώς επιτρέπει διαφορετική σειρά εκτέλεσης εργασιών μεταξύ των μηχανών. Δηλαδή, αν έχουμε n εργασίες και m μηχανές τότε έχουμε $(n!)^m$ λύσεις.

\subsection{Job Shop Scheduling Problem (JSSP)}
Στο συγκεκριμένο πρόβλημα,οι εργασίες δεν απαιτείται να περάσουν από όλες τις μηχανές. Δηλαδή, κάθε εργασία μπορεί να έχει διαφορετική ακολουθία μηχανών και χρόνων επεξεργασίας.Παρόλο που και σε αυτή την περίπτωση υπάρχουν $(n!)^m$ πιθανές λύσεις, το JSSP αποτλείένα πιο πολύπλοκο πρόβλημα από τα προβλήματα Flow-Shop λόγω της μεταβλητής ακολουθίας εργασιών.

\subsection{Open-Shop Scheduling Problem (OSSP)}
Σε αυτό το πρόβλημα, δεν υπάρχει συγκεκριμένη σειρά επεξεργασίας για τις εργασίες στις μηχανές.
Κάθε εργασία μπορεί να εκτελεστεί σε οποιαδήποτε μηχανή, με οποιαδήποτε σειρά. Παρέχει μεγάλη ευελιξία αλλά μπορεί να οδηγήσει σε περίπλοκα προβλήματα δρομολόγησης.

\subsection * {NP-hard προβλήματα}
Τα παραπάνω προβλήματα χρονοπρογραμματισμού θεωρούνται NP-hard. Αυτό σημαίνει ότι ανήκουν στην κατηγορία των προβλημάτων για τα οποία δεν είναι γνωστό να υπάρχει αλγόριθμος που να μπορεί να βρίσκει τη βέλτιστη λύση σε πολυωνυμικό χρόνο για όλες τις περιπτώσεις του προβλήματος. Ένας αλγόριθμος πολυωνυμικού χρόνου έχει χρόνο εκτέλεσης $o(n)^k$ όπου k είναι μία (θετική) σταθερά. Στην πράξη, δηλαδή σε μεγάλες εκδοχές αυτών των προβλημάτων, η εύρεση της απόλυτα βέλτιστης λύσης μπορεί να είναι εξαιρετικά δύσκολη και χρονοβόρα. Συνήθως, για την προσέγγιση αυτών των προβλημάτων χρησιμοποιούνται ευρετικές μέθοδοι ή αλγόριθμοι προσομοίωσης για να βρεθεί μια αποδεκτή λύση σε λογικό χρόνο, αν και αυτή η λύση δεν εγγυάται πάντα ότι είναι η καλύτερη δυνατή.



\section{JSSP problem}
Η δομή που επέλεξα για το διάβασμα των περιεχομένων των αρχείων είναι ένα λεξικό (dictionary) με κλειδί τη διαδρομή του αρχείου. Σαν value περιέχονται υπο-λεξικά, με αποτέλεσμα να υπάρχει διάσπαση των δεδομένων σε βαθμίδες, με το κύριο λεξικό να περιέχει τα διάφορα αρχεία και κάθε αρχείο να περιέχει τα δεδομένα του.



\section{SPT Dispaching rule}
\vspace{0.2cm}

\subsection{Dispaching rules}
Ένας τρόπος επίλυσης ενός JSSP προβλήματος, είναι η χρήσηn ενός dispatching rule. Γενικά οι dispatching rules είναι εύκολοι στην κατανόηση και εφαρμογή αλλά δεν οδηγούν συνήθως στην βέλτιστη λύση.Γενικότερα προτιμούνται πιο προηγμένες τεχνικές όπως ευρετικοί αλγόριθμοι ή προσεγγίσεις μηχανικής μάθησης αν και είναι πιο περίπλοκες και υπολογιστικά απαιτητικές.

\subsection{Shortest Processing Time (SPT)}
Ο κανόνας ανάθεσης Shortest Processing Time (SPT) είναι ένας δημοφιλής κανόνας διαχείρισης στην παραγωγή και τη δρομολόγηση εργασιών. Βασίζεται στην αρχή της επιλογής της εργασίας με τον συνολικό συντομότερο χρόνο επεξεργασίας για να εκτελεστεί πρώτη.

\subsection{Αποτελέσματα SPT}

\begin{tabular}{ l c c c }
\hline
Αρχείο & Υπολογισμένο Makespan (SPT) & Βέλτιστο Makespan & Διαφορά \\
\hline
la01.txt & 1923 & 666 & 1257 \\
la02.txt & 1704 & 655 & 1049 \\
la03.txt & 1632 & 590 & 1042 \\
la04.txt & 1829 & 570 & 1259 \\
la05.txt & 1700 & 593 & 1107 \\
mt06.txt & 143 & 55 & 88 \\
mt10.txt & 2775 & - & - \\
mt20.txt & 3029 & - & - \\
\hline
\end{tabular}



\section{Διάγραμμα Gaant}

Το διάγραμμα Gantt είναι ένας τύπος οριζόντιου χρονοδιαγράμματος που χρησιμοποιείται για διαχείριση έργου. Αυτό το διάγραμμα βοηθά στην απεικόνιση του προγραμματισμού και της προόδου των διάφορων εργασιών ή φάσεων ενός έργου σε έναν χρονικό άξονα.Σε python η υλοποίηση του γίνεται μέσω της βιβλιοθήκης matplotlib.



\section{Αλγόριθμος shifting bottleneck}
\vspace{0.2cm}

\subsection{Εισαγωγή στον αλγόριθμο}
Ο αλγόριθμος Shifted Bottleneck είναι μια ευριτική μέθοδος που χρησιμοποιείται για την επίλυση προβλημάτων Job Shop Scheduling (JSSP). Στοχεύει να βελτιστοποιήσει το makespan, δηλαδή τον συνολικό χρόνο που απαιτείται για να ολοκληρωθούν όλες οι εργασίες.

\subsection{Μέθοδος υλοποίησης}
Ο αλγόριθμος εντοπίζει και βελτιστοποιεί τα bottlenecks σε σειρά εργασιών.Το "bottleneck" σε ένα πρόβλημα δρομολόγησης (όπως το JSSP) αναφέρεται στο στοιχείο της διαδικασίας που περιορίζει τη συνολική απόδοση ή την ταχύτητα του συστήματος. Συνήθως, είναι ένα στάδιο παραγωγής ή μια πόρος που λειτουργεί στο χαμηλότερο ρυθμό σε σχέση με τα υπόλοιπα, δημιουργώντας ένα σημείο συμφόρησης. Η διαδικασία περιλαμβάνει την αναγνώριση του πιο επιβαρυμένου μηχανήματος ή σταδίου, τη δρομολόγηση του και την επαναληπτική αναθεώρηση των bottlenecks.

\end{document}
