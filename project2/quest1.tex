\documentclass{article}
\usepackage[utf8]{inputenc}
\usepackage{amsmath}
\usepackage[greek,english]{babel}
\usepackage{alphabeta}

\title{%
   \normalsize \textsc{Πανεπιστήμιο Ιωαννίνων, Τμήμα Πληροφορικής και Τηλεπικοινωνίων} \\ [1em]
   \normalsize \textsc{Αλγόριθμοι και Πολυπλοκότητα} \\ [1em]
   \vspace{0.5cm}
   \hrule
   \vspace{0.4cm}
   \huge Job shop scheduling problem
   \vspace{0.5cm}
   \hrule
   \vspace{0.5cm}
}

\author{%
   \normalfont
   Κωνσταντίνος Ράντος\\
   \texttt{int02565@uoi.gr}  
}

\date{}

\begin{document}
\maketitle


\vspace{0.4cm}
\begin{center}
\Large Διάφορα προβλήματα χρονοπρογραμμάτισμου.
\end{center}



\section {Permutation Flow-Shop Scheduling Problem (PFSSP) }
Σε αυτό το πρόβλημα, υπάρχουν πολλαπλές μηχανές και εργασίες. Κάθε εργασία πρέπει να εκτελεστεί σε κάθε μηχανή σε συγκεκριμένη σειρά. Η κύρια περιοριστική συνθήκη είναι ότι η σειρά των εργασιών πρέπει να είναι ίδια σε όλες τις μηχανές.Αν έχουμε n μηχανές τότε μπορούμε να έχουμε n! λύσεις.

\section{Flow-Shop Scheduling Problem (FSSP)}
Αυτό το πρόβλημα αφορά επίησης πολλαπλές μηχανές και εργασίες,οι οποίες πρέπει να εκτελεστούν σε κάθε μηχανή σε συγκεκριμένη σειρά. Διαφέρει από το PFSSP καθώς επιτρέπει διαφορετική σειρά εκτέλεσης εργασιών μεταξύ των μηχανών. Δηλαδή, αν έχουμε n εργασίες και m μηχανές τότε έχουμε $(n!)^m$ λύσεις.

\section{Job Shop Scheduling Problem (JSSP)}
Στο συγκεκριμένο πρόβλημα,οι εργασίες δεν απαιτείται να περάσουν από όλες τις μηχανές. Δηλαδή, κάθε εργασία μπορεί να έχει διαφορετική ακολουθία μηχανών και χρόνων επεξεργασίας.Παρόλο που και σε αυτή την περίπτωση υπάρχουν $(n!)^m$ πιθανές λύσεις, το JSSP αποτλείένα πιο πολύπλοκο πρόβλημα από τα προβλήματα Flow-Shop λόγω της μεταβλητής ακολουθίας εργασιών.

\section{Open-Shop Scheduling Problem (OSSP)}
Σε αυτό το πρόβλημα, δεν υπάρχει συγκεκριμένη σειρά επεξεργασίας για τις εργασίες στις μηχανές.
Κάθε εργασία μπορεί να εκτελεστεί σε οποιαδήποτε μηχανή, με οποιαδήποτε σειρά. Παρέχει μεγάλη ευελιξία αλλά μπορεί να οδηγήσει σε περίπλοκα προβλήματα δρομολόγησης.

\section * {NP-hard προβλήματα}
Τα παραπάνω προβλήματα χρονοπρογραμματισμού θεωρούνται NP-hard. Αυτό σημαίνει ότι ανήκουν στην κατηγορία των προβλημάτων για τα οποία δεν είναι γνωστό να υπάρχει αλγόριθμος που να μπορεί να βρίσκει τη βέλτιστη λύση σε πολυωνυμικό χρόνο για όλες τις περιπτώσεις του προβλήματος. Ένας αλγόριθμος πολυωνυμικού χρόνου έχει χρόνο εκτέλεσης $o(n)^k$ όπου k είναι μία (θετική) σταθερά. Στην πράξη, δηλαδή σε μεγάλες εκδοχές αυτών των προβλημάτων, η εύρεση της απόλυτα βέλτιστης λύσης μπορεί να είναι εξαιρετικά δύσκολη και χρονοβόρα. Συνήθως, για την προσέγγιση αυτών των προβλημάτων χρησιμοποιούνται ευρετικές μέθοδοι ή αλγόριθμοι προσομοίωσης για να βρεθεί μια αποδεκτή λύση σε λογικό χρόνο, αν και αυτή η λύση δεν εγγυάται πάντα ότι είναι η καλύτερη δυνατή.

\end{document}
